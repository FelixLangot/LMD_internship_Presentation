\documentclass[10pt]{beamer}

\usetheme[progressbar=frametitle]{metropolis}
\usepackage{appendixnumberbeamer}

\usepackage{booktabs}
\usepackage[scale=2]{ccicons}

\usepackage{pgfplots}
\usepgfplotslibrary{dateplot}

\usepackage{xspace}
\newcommand{\themename}{\textbf{\textsc{metropolis}}\xspace}

\usepackage{caption}
%\usepackage{gensymb}
\usepackage{ulem}
\usepackage{subcaption}
\usepackage[french]{babel}
\usepackage{csquotes}
\usepackage[backend=biber,style=authoryear,citestyle=authoryear]{biblatex}

\DeclareCaptionLabelFormat{underlcap}{\uline{#1 #2}}
\DeclareCaptionLabelSeparator{underlcap}{~}
\DeclareCaptionTextFormat{underlcap}{\expandafter\uline\expandafter{\expandafter#1}}

\captionsetup[figure]{%
   labelformat=underlcap,labelseparator=underlcap,textformat=underlcap}
   
\newcommand\Wider[2][3em]{%
\makebox[\linewidth][c]{%
  \begin{minipage}{\dimexpr\textwidth+#1\relax}
  \raggedright#2
  \end{minipage}%
  }%
}

\makeatletter
\setlength{\metropolis@progressinheadfoot@linewidth}{1.7pt}
\setlength{\metropolis@titleseparator@linewidth}{1.7pt}
\setlength{\metropolis@progressonsectionpage@linewidth}{1.7pt}

\definecolor{alizarin}{rgb}{0.1, 0.26, 0.82}
\definecolor{bazaar}{rgb}{0.63, 0.63, 0.8}

\setbeamercolor{progress bar}{fg=alizarin,bg=bazaar}

\AtBeginSection[]{
  \begin{frame}
  \vfill
  \centering
  \begin{beamercolorbox}[sep=8pt,center,shadow=true,rounded=true]{title}
    \usebeamerfont{title}\insertsectionhead\par%
  \end{beamercolorbox}
  \vfill
  \end{frame}
}

\addbibresource{/Applications/ZoteroBibs/Library.bib}

\title{Concours de l'école doctorale des Sciences de l'Environnement}
\subtitle{Candidat: Félix Langot}
\date{\today}
\author{\textit{UVSQ/Paris-Saclay - University of Bristol}}
\institute{}
% \titlegraphic{\vfill\includegraphics[height=1.5cm]{Figures/logo.png}}
\begin{document}

\maketitle

\section*{Présentation du cursus}
\begin{frame}{\secname}
\Wider{
\textbf{Baccalauréat S (2016):} 
  \begin{itemize}
      \item mention TB, mention européenne, spécialité mathématiques.
      \item 17.5 de moyenne générale dont 19/20 en mathématiques et 17/20 en physique
  \end{itemize} 
\textbf{MSci Physics with Astrophysics (2020):} 
  \begin{itemize}
      \item Obtention du master avec 'Upper second class honours' (mention bien)
      \item 'commendation' pour le projet final de master (note > 16)
      \item Passage de plusieurs unités avec des notes '1st class' (mention très bien) dont l'unité \textit{Geophysical Fluid Dynamics}
  \end{itemize}
\textbf{Master ECLAT (2021):} 
  \begin{itemize}
      \item Moyenne du premier semestre de 15.4/20
      \item 18/20 de moyenne dans les U.E. de modélisation
  \end{itemize}
}
\end{frame}

\section*{Expérience de recherche}
\begin{frame}{\secname}
\Wider{  
  \begin{columns}
  \column{0.7\textwidth}
    \textbf{MSci Physics with Astrophysics:} 
    \begin{itemize}
      \item Mesure de la vitesse de l'expansion de l'Univers $H_0$ en utilisant des observations rayon X de galaxies lointaines et l'effet de Sunyaev-Zel'dovich
      \item Simulations d'allées de tourbillons de Karman avec la méthode Lattice-Boltzmann avec parallélisation des processus
    \end{itemize}
  \column{0.3\textwidth}
    \begin{figure}[hbtp]
      \centering
      \includegraphics[width=3.6cm]{/Users/felixlangot/Library/Mobile Documents/com~apple~CloudDocs/Work/IV. Year 4/Project/Assessed/Final Report/13491regions.png}
\end{figure}
\end{columns}
\begin{columns}
  \column{0.5\textwidth}
  \begin{figure}[hbtp]
    \centering
    \includegraphics[width=5cm]{/Users/felixlangot/Library/Mobile Documents/com~apple~CloudDocs/Work/IV. Year 4/Advanced Computing/Mini-Project/Figure:Gifs/stlRe65.png}
  \end{figure}
  \column{0.5\textwidth}
  \begin{figure}[hbtp]
    \centering
    \includegraphics[width=5.5cm]{/Users/felixlangot/Library/Mobile Documents/com~apple~CloudDocs/Work/IV. Year 4/Project/Assessed/Final Report/HubbleGraphs/Logs.eps}
  \end{figure}
\end{columns}
}
\end{frame}

\begin{frame}{\secname}
\vspace{-0.3cm}
\Wider{
\begin{columns}
  \column{0.5\textwidth}
  \textbf{M2 ECLAT:}
  \begin{itemize}
    \item Stage au LMD: Impact de l'organisation de la convection profonde sur l'humidité de la troposphère \\
    $\rightarrow$ Publication des résultats prévue par Dr C. Risi
  \end{itemize}
  \column{0.5\textwidth}
  \hspace{-1cm}
  \begin{figure}[hbtp]
    \centering
    \includegraphics[width=5.5cm]{/Users/felixlangot/Stage/GitRepos/Codes/Figs/meanRHpvsRHa.png}
  \end{figure}
\end{columns}
\begin{figure}[hbtp]
  \centering
  \includegraphics[width=10cm]{/Users/felixlangot/Stage/GitRepos/Codes/Figs/AllComps.png}
\end{figure}
}
\end{frame}

\section*{Projet}
\begin{frame}{\secname}
\Wider{
\textbf{Contexte:}
\begin{itemize}
  \item Projections climatiques incertaines, principalement à cause des nuages de couche limite
\end{itemize}
\pause
\textbf{But:}
\begin{itemize}
  \item Comprendre le rôle de l'organisation à méso-échelle de ces nuages sur leur rétroaction climatique
  \item Établir des contraintes sur l’amplitude de la rétroaction des nuages bas
\end{itemize}
\pause
\textbf{Moyens:}
\begin{itemize}
  \item \textbf{Apprentissage automatique:} catégoriser les morphologies
  \item \textbf{Observation satellites:} étude de la corrélation entre changements morphologiques et variations de la dynamique de couche limite
  \item \textbf{Simulations haute-résolution:} Implémenter les processus de fine échelle.
\end{itemize}
}
\end{frame}

{\usebackgroundtemplate{\includegraphics[width=128mm]{./Figures/Fowers.jpeg}}
\begin{frame}{Merci pour votre attention}

\end{frame}
}
\end{document}
