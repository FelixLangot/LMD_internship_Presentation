\documentclass[10pt]{beamer}

\usetheme[progressbar=frametitle]{metropolis}
\usepackage{appendixnumberbeamer}

\usepackage{booktabs}
\usepackage[scale=2]{ccicons}

\usepackage{pgfplots}
\usepgfplotslibrary{dateplot}

\usepackage{xspace}
\newcommand{\themename}{\textbf{\textsc{metropolis}}\xspace}

\usepackage{caption}
%\usepackage{gensymb}
\usepackage{ulem}
\usepackage{subcaption}
\usepackage[french]{babel}
\usepackage{csquotes}
\usepackage[backend=biber,style=authoryear,citestyle=authoryear]{biblatex}

\DeclareCaptionLabelFormat{underlcap}{\uline{#1 #2}}
\DeclareCaptionLabelSeparator{underlcap}{~}
\DeclareCaptionTextFormat{underlcap}{\expandafter\uline\expandafter{\expandafter#1}}

\captionsetup[figure]{%
   labelformat=underlcap,labelseparator=underlcap,textformat=underlcap}
   
\newcommand\Wider[2][3em]{%
\makebox[\linewidth][c]{%
  \begin{minipage}{\dimexpr\textwidth+#1\relax}
  \raggedright#2
  \end{minipage}%
  }%
}

\makeatletter
\setlength{\metropolis@progressinheadfoot@linewidth}{1.7pt}
\setlength{\metropolis@titleseparator@linewidth}{1.7pt}
\setlength{\metropolis@progressonsectionpage@linewidth}{1.7pt}

\definecolor{alizarin}{rgb}{0.1, 0.26, 0.82}
\definecolor{bazaar}{rgb}{0.63, 0.63, 0.8}

\setbeamercolor{progress bar}{fg=alizarin,bg=bazaar}

\AtBeginSection[]{
  \begin{frame}
  \vfill
  \centering
  \begin{beamercolorbox}[sep=8pt,center,shadow=true,rounded=true]{title}
    \usebeamerfont{title}\insertsectionhead\par%
  \end{beamercolorbox}
  \vfill
  \end{frame}
}

\addbibresource{/Applications/ZoteroBibs/Library.bib}

\title{Prédire l'humidité troposphérique en fonction de l'organisation de la convection et de la circulation de grande échelle}
\subtitle{Félix Langot}
\date{\today}
\author{\textit{LMD - UVSQ/Paris-Saclay}}
\institute{}
% \titlegraphic{\vfill\includegraphics[height=1.5cm]{Figures/logo.png}}
\begin{document}

\maketitle

\section*{Introduction}
\begin{frame}{\secname}

    \begin{itemize}
        \item \textbf{But du stage:} développer un modèle théorique simple pour quantifier l'effet de l'organisation de la convection et de la circulation de grande échelle sur l'humidité de la troposphère 
        \item Utilisation de simulations CRMs $\rightarrow$ vérifier les hypothèses du modèle + évaluer son réalisme. 
        \item Différentes distributions de l'humidité relative (RH) dans la troposphère, dues à: \\
        $\rightarrow$ l'agrégation de la convection: fait baisser la RH  \\
        $\rightarrow$ l'ascendance: humidifie la troposphère \\
        \autocite{Tobin2012}
    \end{itemize}

\end{frame}

\begin{frame}{\secname}
    \begin{itemize}
        \item \textbf{Cloud-Resolving  Model  (CRM):} Permet de simuler des nuages convectifs avec une résolution allant de la dizaine de mètre à plusieurs kilomètres sur un espace limité en 4D (temps+espace).
        \item \textbf{Simulation en équilibre radiatif-convectif (RCE):} 
        \begin{itemize}
            \item Simulation avec une boîte doublement périodique, typiquement utilisé pour étudier le climat tropical
            \item Fixation de la température de surface + saturation de l'air à la surface comme c'est le cas au-dessus de l'océan. 
            \item Inclusion de l'énergie solaire qui évapore l'eau. 
            \item Redistribution de l'énergie par les équations de la mécanique des fluides, les radiation et la thermodynamique qui génèrent un cycle hydrologique avec des précipitations pour équilibrer l'évaporation.
        \end{itemize}
    \end{itemize}
\end{frame}

\begin{frame}{\secname}
    \begin{itemize}
        \item \textbf{Pourquoi et comment représenter la circulation de grande-échelle:} 
        \begin{itemize}
            \item Impacte l'humidité de l'atmosphère par advection mais aussi l'organisation de la convection \autocite{dufauxRapportStageLicence2021}, qui à son tour modifie la RH. 
            \item Représenter l’ascendance de grande échelle $\rightarrow$ ajout d'un terme d’advection verticale d’humidité et de température.
        \end{itemize}
        \item \textbf{Obtention de différents types d'organisation:} On ajoute au RCE un forçage différent en fonction du type d'organisation que l'on veut favoriser
        \begin{itemize}
            \item Cumulonimbus isolés: RCE
            \item Cyclones: RCE + Coriolis
            \item Ligne de grain: RCE + cisaillement de vent
        \end{itemize}
    \end{itemize}

\end{frame}

\begin{frame}{\secname}

    \begin{itemize}
        \item Effets vérifiés par le CRM SAM, avec lequel on peut calculer la RH simulée d'une parcelle troposphérique sur un domaine de 95x95 km
        $$
        RH_{actual} = \left.\frac{q_v}{q_{sat}}\right|_{z_{parcel}=5km}
        $$
        avec $q_v$ l'humidité spécifique et $q_{sat}$ l'humidité spécifique de saturation, en g/kg.
    \end{itemize}

\end{frame}

\begin{frame}{\secname}
    
    \begin{figure}
        \centering
        \includegraphics[width=9cm]{../Codes/Figs/RHaProbsSuperposition.png}
        \label{RHactual}
    \end{figure}

\end{frame}

\section*{Comment prédire les distributions de \textit{RH}?}
\begin{frame}{\secname}
     \begin{itemize}
         \item Pour tenter de prédire les distributions de RH, on utilise le modèle d'advection-condensation \autocite{pierrehumbertRelativeHumidityAtmosphere2007,Vallis2017}
     \end{itemize}
     \begin{figure}[hbtp]
         \centering
         \includegraphics[width=6.6cm]{Figures/advec-condens.png}
     \end{figure}
\end{frame}

\begin{frame}{\secname}
    $\rightarrow$ L'humidité spécifique d'une parcelle $q_v$ reste constante en l'absence de saturation (et donc de condensation) \\ \vspace{0.2cm}
    $\rightarrow$ On dispose de l'humidité spécifique de saturation $q_{sat}$ en fonction de l'altitude grâce au modèle SAM  \\ \vspace{0.2cm}
    $\rightarrow$ On peut donc prédire la RH de la troposphère à partir de l'altitude de dernière saturation de la parcelle
    \begin{figure}[hbtp]
        \centering
        \includegraphics[width=6cm]{Figures/lsadraw.png}
    \end{figure}
\end{frame}

\begin{frame}{\secname}
    Comment le modèle met en évidence les relations entre la RH et la circulation de grande échelle/l'organisation de la convection? \\ \vspace{1cm}
    $\rightarrow$ Ascendance $\rightarrow$ vitesse de subsidence de l'environnment $\rightarrow$ temps de descente de la parcelle. Si temps de descente augmente, la probabilité de rencontrer un nuage augmente. \\ \vspace{1cm}
    $\rightarrow$ Organisation $\rightarrow$ distribution des nuages $\rightarrow$ probabilité de rencontre. Nuages sont très regroupés = baisse de la probabilité
\end{frame}

\section*{Altitude de dernière saturation - Approche statique}
\begin{frame}{\secname}
    Comment trouver l'altitude de dernière saturation d'une parcelle se situant dans la troposphère?
    \begin{itemize}
        \item \textbf{Hypothèse statique:} L'altitude de dernière saturation d'une parcelle troposphérique correspond à l'altitude du nuage le plus proche au-dessus de la parcelle.
        \item SAM: rapports de mélange d'eau et de glace (\textit{water mixing ratio, ice mixing ratio}) $q_c, q_i$ \\ 
        $\rightarrow$ Détection des nuages
        \begin{itemize}
            \item Si $q_c + q_i > 10^{-6}$ alors le point de grille est dans un nuage \autocite{Risi2021}
        \end{itemize}
    \end{itemize}
    \begin{figure}[hbtp]
        \centering
        \includegraphics[width=5cm]{../Codes/Figs/lastsaturationlight.png}
    \end{figure}
\end{frame}

\begin{frame}{\secname}
    On peut donc mesurer l'altitude du nuage le plus proche de chaque point de grille à chaque pas de temps des simulations
    \begin{figure}[hbtp]
        \centering
        \includegraphics[width=10cm]{../Codes/Figs/3Dclouds.png}
    \end{figure}
\end{frame}

\begin{frame}{\secname}
    L'humidité relative prédite RH$_p$ peut ensuite être calculée en fonction de $q_{sat}$ seulement:
    $$
    RH_p = \frac{q_{sat}(z_{clouds})}{q_{sat}(z_{parcel})}
    $$
    où $z_{clouds}$ est l'altitude des nuages les plus proches de la troposphère au-dessus des points de grille à $z_{parcel} = 5km$, l'altitude choisie dans la troposphère
\end{frame}

\begin{frame}{\secname}
    On peut comparer la distribution obtenue avec cette méthode à la distribution réelle de la RH


    \Wider{
        \begin{figure}[hbtp]
            \centering
            \includegraphics[width=12cm]{../Codes/Figs/RhvsRHp.png}
        \end{figure}
    $\rightarrow$ La RH ne peut pas être prédite par un modèle statique, l'intermittence des nuages et le mouvement de la parcelle sont importants.
        }
\end{frame}

\section*{Altitude de dernière saturation - Approche dynamique}

\begin{frame}{\secname}
    \Wider{
\begin{columns}
    \column{0.4\textwidth}
    \begin{itemize}
        \item \textbf{Approche dynamique:} On considère le mouvement vertical d'une parcelle au-dessus de la troposphère.
        \item SAM fournit la vitesse verticale $w$, depuis laquelle on peut déduire $w_{env}$, la vitesse en dehors des nuages. 
    \end{itemize}
    \column{0.7\textwidth}
    \begin{figure}[hbtp]
        \centering
        \includegraphics[width=7cm]{../Codes/Figs/lastsaturationsubsidencelight.png}
    \end{figure}
\end{columns}
    }
\end{frame}

\begin{frame}{\secname}
    Simulations avec ascendance: vitesse verticale totale dans l'environnement $w_{tot} = w_{env} + w_{LS}$, où $w_{LS}$ est l'ascendance imposée. \\
    Vitesse totale des parcelles dans l'environnement:
    
    \begin{figure}[hbtp]
        \centering
        \includegraphics[width=7cm]{../Codes/Figs/wtotz.png}
    \end{figure}
\end{frame}

\begin{frame}{\secname}
    \begin{itemize}
        \item On se concentre premièrement sur la simulation de cumulonimbus sans ascendance
        \item On choisit 10 parcelles étant à l'altitude $z_{parcel}$ aux 10 derniers pas de temps. 
        \item On remonte le temps et on trace la trajectoire de la parcelle, gouvernée par $w_{tot}$.
    \end{itemize}
    \begin{figure}[hbtp]
        \centering
        \includegraphics[width=7.5cm]{../Codes/Figs2/ztraj.png}
    \end{figure}
\end{frame}

\begin{frame}{\secname}
    En utilisant les mêmes conditions pour détecter les nuages, on obtient les altitudes de dernières saturations suivantes:
    \begin{figure}[hbtp]
        \centering
        \includegraphics[width=10cm]{../Codes/Figs2/zclouds_traj.png}
    \end{figure}
\end{frame}

\begin{frame}[allowframebreaks]{Bibliographie}
    \printbibliography
\end{frame}


\end{document}
