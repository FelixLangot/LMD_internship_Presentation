\documentclass[10pt]{beamer}

\usetheme[progressbar=frametitle]{metropolis}
\usepackage{appendixnumberbeamer}

\usepackage{booktabs}
\usepackage[scale=2]{ccicons}

\usepackage{pgfplots}
\usepgfplotslibrary{dateplot}

\usepackage{xspace}
\newcommand{\themename}{\textbf{\textsc{metropolis}}\xspace}

\usepackage{caption}
%\usepackage{gensymb}
\usepackage{ulem}
\usepackage{subcaption}
\usepackage[french]{babel}
\usepackage{csquotes}
\usepackage[backend=biber,style=authoryear,citestyle=authoryear]{biblatex}

\DeclareCaptionLabelFormat{underlcap}{\uline{#1 #2}}
\DeclareCaptionLabelSeparator{underlcap}{~}
\DeclareCaptionTextFormat{underlcap}{\expandafter\uline\expandafter{\expandafter#1}}

\captionsetup[figure]{%
   labelformat=underlcap,labelseparator=underlcap,textformat=underlcap}
   
\newcommand\Wider[2][3em]{%
\makebox[\linewidth][c]{%
  \begin{minipage}{\dimexpr\textwidth+#1\relax}
  \raggedright#2
  \end{minipage}%
  }%
}

\makeatletter
\setlength{\metropolis@progressinheadfoot@linewidth}{1.7pt}
\setlength{\metropolis@titleseparator@linewidth}{1.7pt}
\setlength{\metropolis@progressonsectionpage@linewidth}{1.7pt}

\definecolor{alizarin}{rgb}{0.1, 0.26, 0.82}
\definecolor{bazaar}{rgb}{0.63, 0.63, 0.8}

\setbeamercolor{progress bar}{fg=alizarin,bg=bazaar}

\AtBeginSection[]{
  \begin{frame}
  \vfill
  \centering
  \begin{beamercolorbox}[sep=8pt,center,shadow=true,rounded=true]{title}
    \usebeamerfont{title}\insertsectionhead\par%
  \end{beamercolorbox}
  \vfill
  \end{frame}
}

\addbibresource{/Applications/ZoteroBibs/Library.bib}

\title{Prédire l'humidité troposphérique en fonction de l'aggrégation de la convection et de l'ascendance}
\subtitle{Félix Langot}
\date{\today}
\author{\textit{LMD - UVSQ/Paris-Saclay}}
\institute{}
% \titlegraphic{\vfill\includegraphics[height=1.5cm]{Figures/logo.png}}
\begin{document}

\maketitle

\section*{Introduction}
\begin{frame}{\secname}

    \begin{itemize}
        \item Différentes distributions de l'humidité relative (RH) dans la troposphère, dues à: \\
        $\rightarrow$ l'aggrégation de la convection \\
        $\rightarrow$ l'ascendance
        \item L'aggrégation fait baisser la RH 
        \item L'ascendance humidifie la troposphère
        \item Vérifié avec le CRM SAM, avec lequel on peut calculer la RH réelle d'une parcelle troposphérique sur un domaine de 95x95 km
        $$
        RH_{actual} = \left.\frac{q_v}{q_{sat}}\right|_{z_{parcel}=5km}
        $$
        avec $q_v$ l'humidité spécifique et $q_{sat}$ l'humidité spécifique de saturation, en g/kg.
    \end{itemize}

\end{frame}

\begin{frame}{\secname}
    
    \begin{figure}
        \centering
        \includegraphics[width=9cm]{../Codes/Figs/RHaProbsSuperposition.png}
        \label{RHactual}
    \end{figure}

\end{frame}

\section*{Comment prédire les distributions de \textit{RH}?}
\begin{frame}{\secname}
     \begin{itemize}
         \item Pour tenter de prédire les distributions de RH, on utilise le modèle d'advection-condensation \autocite{pierrehumbertRelativeHumidityAtmosphere2007,Vallis2017} \\ \vspace{0.2cm}
         $\rightarrow$ L'humidité spécifique d'une parcelle $q_v$ reste constante en l'absence de saturation (et donc de condensation) \\ \vspace{0.2cm}
         $\rightarrow$ On dispose de l'humidité spécifique de saturation $q_{sat}$ en fonction de l'altitude grâce au modèle SAM  \\ \vspace{0.2cm}
         $\rightarrow$ On peut donc prédire la RH de la troposphère à partir de l'altitude de dernière saturation de la parcelle. 
         \item Deux approches essayées jusqu'à présent: statique et dynamique simplifiée 
     \end{itemize}
\end{frame}

\section*{Altitude de dernière saturation - Approche statique}
\begin{frame}{\secname}
    Comment trouver l'altitude de dernière saturation d'une parcelle se situant dans la troposphère?
    \begin{itemize}
        \item \textbf{Hypothèse statique:} L'altitude de dernière saturation d'une parcelle troposphérique correspond à l'altitude du nuage le plus proche au-dessus de la parcelle.
        \item SAM: rapports de mélange d'eau et de glace (\textit{water mixing ratio, ice mixing ratio}) $q_c, q_i$ \\ 
        $\rightarrow$ Détection des nuages
        \begin{itemize}
            \item Si $q_c + q_i > 10^{-6}$ alors le point de grille est dans un nuage \autocite{Risi2021}
        \end{itemize}
    \end{itemize}
    \begin{figure}[hbtp]
        \centering
        \includegraphics[width=5cm]{../Codes/Figs/lastsaturationlight.png}
    \end{figure}
\end{frame}

\begin{frame}{\secname}
    On peut donc mesurer l'altitude du nuage le plus proche de chaque point de grille à chaque pas de temps des simulations
    \begin{figure}[hbtp]
        \centering
        \includegraphics[width=10cm]{../Codes/Figs/3Dclouds.png}
    \end{figure}
\end{frame}

\begin{frame}{\secname}
    L'humidité relative prédite RH$_p$ peut ensuite être calculée en fonction de $q_{sat}$ seulement:
    $$
    RH_p = \frac{q_{sat}(z_{clouds})}{q_{sat}(z_{parcel})}
    $$
    où $z_{clouds}$ est l'altitude des nuages les plus proches de la troposphère au-dessus des points de grille à $z_{parcel} = 5km$, l'altitude choisie dans la troposphère
\end{frame}

\begin{frame}{\secname}
    On peut comparer la distribution obtenue avec cette méthode à la distribution réelle de la RH


    \Wider{
        \begin{figure}[hbtp]
            \centering
            \includegraphics[width=12cm]{../Codes/Figs/RhvsRHp.png}
        \end{figure}
    $\rightarrow$ La RH ne peut pas être prédite par un modèle statique.
        }
\end{frame}

\section*{Altitude de dernière saturation - Approche dynamique}

\begin{frame}{\secname}
    \Wider{
\begin{columns}
    \column{0.4\textwidth}
    \begin{itemize}
        \item \textbf{Approche dynamique:} On considère le mouvement vertical d'une parcelle au-dessus de la troposphère.
        \item SAM fournit la vitesse verticale $w$, depuis laquelle on peut déduire $w_{env}$, la vitesse en dehors des nuages. 
    \end{itemize}
    \column{0.7\textwidth}
    \begin{figure}[hbtp]
        \centering
        \includegraphics[width=7cm]{../Codes/Figs/lastsaturationsubsidencelight.png}
    \end{figure}
\end{columns}
    }
\end{frame}

\begin{frame}{\secname}
    \centering
    \includegraphics[width=9cm]{../Codes/Figs/wenvz.png}
\end{frame}

\begin{frame}{\secname}
    Simulations avec ascendance: vitesse verticale totale dans l'environnement $w_{tot} = w_{env} + w_{LS}$, où $w_{LS}$ est l'ascendance imposée. \\
    Vitesse totale des parcelles dans l'environnement:
    
    \begin{figure}[hbtp]
        \centering
        \includegraphics[width=7cm]{../Codes/Figs/wtotz.png}
    \end{figure}
\end{frame}

\begin{frame}{\secname}
    \begin{itemize}
        \item On se concentre premièrement sur la simulation de cumulonimbus sans ascendance
        \item Le pas de temps est réduit à 30min
        \item Le profil de vitesse est le suivant
    \end{itemize}
    \begin{figure}[hbtp]
        \centering
        \includegraphics[width=8.5cm]{../Codes/Figs2/wmu.png}
    \end{figure}
\end{frame}

\begin{frame}{\secname}
    \begin{itemize}
        \item On choisit 10 parcelles étant à l'altitude $z_{parcel}$ aux 10 derniers pas de temps. 
        \item On remonte le temps et on trace la trajectoire de la parcelle, gouvernée par $w_{\mu}$.
    \end{itemize}
    \begin{figure}[hbtp]
        \centering
        \includegraphics[width=7.5cm]{../Codes/Figs2/ztraj.png}
    \end{figure}
\end{frame}

\begin{frame}{\secname}
    \begin{itemize}
        \item La trajectoire est discrétisée pour correspondre aux altitudes du modèle et ainsi permettre d'associer des valeurs $q$ à chaque altitude traversée pa la parcelle. 
        \item La trajectoire discrétisée sous-estime systématiquement l'altitude de la parcelle
    \end{itemize}
    \begin{figure}[hbtp]
        \centering
        \includegraphics[width=9cm]{../Codes/Figs2/Discrete_traj.png}
    \end{figure}
\end{frame}

\begin{frame}{Bibliographie}
    \printbibliography
\end{frame}


\end{document}
