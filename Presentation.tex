\documentclass[10pt]{beamer}

\usetheme[progressbar=frametitle]{metropolis}
\usepackage{appendixnumberbeamer}

\usepackage{booktabs}
\usepackage[scale=2]{ccicons}

\usepackage{pgfplots}
\usepgfplotslibrary{dateplot}

\usepackage{xspace}
\newcommand{\themename}{\textbf{\textsc{metropolis}}\xspace}

\usepackage{caption}
%\usepackage{gensymb}
\usepackage{ulem}
\usepackage{subcaption}
\usepackage[french]{babel}
\usepackage{csquotes}
\usepackage[backend=biber,style=authoryear,citestyle=authoryear]{biblatex}

\DeclareCaptionLabelFormat{underlcap}{\uline{#1 #2}}
\DeclareCaptionLabelSeparator{underlcap}{~}
\DeclareCaptionTextFormat{underlcap}{\expandafter\uline\expandafter{\expandafter#1}}

\captionsetup[figure]{%
   labelformat=underlcap,labelseparator=underlcap,textformat=underlcap}
   
\newcommand\Wider[2][3em]{%
\makebox[\linewidth][c]{%
  \begin{minipage}{\dimexpr\textwidth+#1\relax}
  \raggedright#2
  \end{minipage}%
  }%
}

\makeatletter
\setlength{\metropolis@progressinheadfoot@linewidth}{1.7pt}
\setlength{\metropolis@titleseparator@linewidth}{1.7pt}
\setlength{\metropolis@progressonsectionpage@linewidth}{1.7pt}

\definecolor{alizarin}{rgb}{0.1, 0.26, 0.82}
\definecolor{bazaar}{rgb}{0.63, 0.63, 0.8}

\setbeamercolor{progress bar}{fg=alizarin,bg=bazaar}

\AtBeginSection[]{
  \begin{frame}
  \vfill
  \centering
  \begin{beamercolorbox}[sep=8pt,center,shadow=true,rounded=true]{title}
    \usebeamerfont{title}\insertsectionhead\par%
  \end{beamercolorbox}
  \vfill
  \end{frame}
}

\addbibresource{/Applications/ZoteroBibs/Library.bib}

\title{Prédire l'humidité troposphérique en fonction de l'aggrégation de la convection et de l'ascendance}
\subtitle{Félix Langot}
\date{\today}
\author{\textit{LMD - UVSQ/Paris-Saclay}}
\institute{}
% \titlegraphic{\vfill\includegraphics[height=1.5cm]{Figures/logo.png}}
\begin{document}

\maketitle

\section*{Introduction}
\begin{frame}{\secname}
    \begin{columns}
        \column{0.4\textwidth}
            \begin{itemize}
                \item Différentes distributions de l'humidité relative (RH) dues à: \\
                $\rightarrow$ l'aggrégation de la convection \\
                $\rightarrow$ l'ascendance
            \end{itemize}
        \column{0.8\textwidth}
        \begin{figure}
            \centering
            \includegraphics[width=8cm]{../Codes/Figs/RHaProbsSuperposition.png}
            \label{RHactual}
        \end{figure}
    \end{columns}
\end{frame}

\begin{frame}{\secname}
    \begin{columns}
        \column{0.42\textwidth}
            \begin{itemize}
                \item L'aggrégation fait baisser la RH \item L'ascendance humidifie la troposphère
            \end{itemize}
        \column{0.75\textwidth}
        \begin{figure}
            \centering
            \includegraphics[width=8cm]{../Codes/Figs/RHaProbsSuperposition.png}
        \end{figure}
    \end{columns}
\end{frame}

\section*{But}
\begin{frame}{\secname}
     \begin{itemize}
         \item Pour tenter de prédire les distributions de RH, on utilise le paradigme de \textit{last saturation} \autocite{sherwoodMaintenanceFreeTroposphericTropical1996}
     \end{itemize}
\end{frame}

\begin{frame}{Bibliographie}
    \printbibliography
\end{frame}


\end{document}
